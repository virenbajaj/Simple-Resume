\documentclass[11pt, letterpaper]{simple_resume}[2020/07/29]

\begin{document}

% ---------------------------------------------------------
%                       Header + Footer
%----------------------------------------------------------
\makeheader{Viren Bajaj} %name
           {10 Kasch Ct, Monroe, NY 10950} %address
           {(412)-737-4028} %phone
           {viren.bajaj@columbia.edu} %email
           {viren-bajaj} %linkedin
           {virenbajaj} %github

\makefooter{\today}
          {Viren Bajaj · Résumé}
          {\thepage}
% ---------------------------------------------------------
%                       Education
%----------------------------------------------------------
\section{Education}
% \locatedsubsection{Carnegie Mellon University}{Pittsburgh, PA}
% \datedsubsection{B.S. in Physics (Computer Science Minor)}{May 2018}
% {\textit{B.S. in Physics (Computer Science Minor)} \hfill May 2018}
\educationentry{Columbia University} % University
               {New York, NY} % Location
               {M.S. in Computer Science (Machine Learning)} %degree
               {Expected Dec 2021} % Graduation month and year
               {Advanced Machine Learning} % Courses
\vspace{-1ex}               
\educationentry[Mellon College of Science Research Honors; Dean's List,                   High Honors (Spring'16); Dean's List (Spring'18)]                       %optional honors
               {Carnegie Mellon University} %university
               {Pittsburgh, PA} %location
               {B.S. in Physics (Minor in Computer Science); GPA: 3.42} %degree
               {May 2018} %graduation month and year
               {Machine Learning (Graduate Level), Cognitive Robotics, Practical Data Science, Cyber Physical Systems} % courses

% ---------------------------------------------------------
%                       Skills
%----------------------------------------------------------
\section{Technical Skills}

\begin{skills}
    \skill  {Languages}  {Python, MATLAB, C / C++}
    \skill {Frameworks}  {TensorFlow, OpenCV, Jupyter, Pandas, GeoPandas,                             Matplotlib, seaborn, Folium, Flask}
    \skill {Databases}   {MySQL, PostgreSQL, Overpass API}
    \skill {Other Tools} {Git, \TeX, GNURadio, Mission Planner, ROS, Aerostack,                             ROOT}
\end{skills}

% ---------------------------------------------------------
%                       Experience
%----------------------------------------------------------
\section{Experience}
% \locatedsubsection{NASA Langley Research Center {[National Institute of Aerospace]}} {Hampton, VA}
\experienceentry{NASA Langley Research Center {[National Institute                  of Aerospace]}} % Employer
                {Hampton, VA} % Location
                {Associate Research Engineer} %Position
                {Oct 2018 - Jul 2020} % Time
{\begin{items}
    \item {Conceptualized a Radio Frequency Interference (RFI) service for UAS to avoid RF hazards in urban airspace with a cross-functional team of ten researchers (\textit{papers to be published soon})}
    \item {Co-led \$300K sensor network  acquisition and install after building a prototype service by performing signal processing with GNURadio and creating Radio Environment Maps with GeoPandas, PyKrige, and Folium}
    \item {Developed online RF sensing algorithms: modulation classifier in TensorFlow, RSSI predictor in MATLAB }
    \item {Co-developed a reinforcement learning based guidance selector for a UAS sense-and-avoid algorithm; published research paper in International Conference on Dependable, Autonomic and Secure Computing 2019}
\end{items}}

\vspace{-1ex}

\experienceentry{FasicuChain}{Dubai, UAE}{Co-founder}{May 2018 - Sept 2018}
{\begin{items}
    \item {Generated and pitched idea, sales, and investment decks with three co-founders of FasicuChain - an \\ 
    anti-counterfeiting track-and-trace solution featuring uncloneable labels and blockchain} 
    \item {Managed an agile team of three to develop FasicuChain Android application}
\end{items}
}
\vspace{-1ex}
% \datedsubsection{Visiting Student}{May 2017 - Jul 2017}
% {\begin{items}

%     \item {Built a testbed to study failure modes of machine learning based planning algorithms by equipping a rover with a Pixhawk, PixyCam, and Aerostack - a ROS/C++ based modular framework autonomous vehicles}
%     % \item {Built a test-bed to study failure modes of motion planning algorithms based on machine learning}
%     % \item {Installed \href{https://github.com/Vision4UAV/Aerostack/wiki/What-is-Aerostack}{Aerostack}, a modular framework for autonomous aerial robotic systems, using ROS/C++ on a hpi-racing e-buggy and equipped it with a Pixhawk autopilot and PixyCam vision sensor}
% \end{items}}

\experienceentry{Department of Physics, Carnegie Mellon University} % Employer
                {Pittsburgh, PA} % Location
                {Research Assistant to Prof. Reinhard Schumacher} %Position
                {May 2016 - May 2017} % Time
{\begin{items}
    \item {Discovered reaction mechanism of proton antiproton photo-production to be t-channel dominated by performing statistical analyses in ROOT/C++ on data from the Thomas Jefferson National Accelerator Facility}
    \item {Presented poster at the 2017 Annual Meeting of Division of Nuclear Physics, American Physical Society}
\end{items}}

% ---------------------------------------------------------
%                       Projects
%----------------------------------------------------------
\section{Projects}
\experienceentry{Verifiably Safe SCUBA Diving using Commodity Sensors}%Employer
                {Pittsburgh, PA} % Location
                {Research advised by Prof. André Platzer, Carnegie Mellon University} %Position
                {Jan 2017 - Oct 2018} % Time
{\begin{items}
    % \item {Created a SCUBA diving computer program which increases dive time and reduces risk of decompression sickness by using commodity heart rate sensors - modeled in differential dynamic logic and verified using KeYmaeraX theorem prover}
    \item{Created and verified a novel model and control algorithm for diving, preventing need for a hose or wireless transmitter by indirectly estimating a diver’s oxygen consumption using commodity heart rate sensors}
    \item {Published research paper in The International Conference on Embedded Software 2019 (WiP Track);\\ presentation won second prize at Verification and Validation Grand-Prix 2016 at Carnegie Mellon University}
\end{items}}
% \section{Interests}
% {Basketball, Bouldering}
\end{document}
